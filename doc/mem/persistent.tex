\section{Sketch}
\begin{outline}
\1 initial definitions
  \2 \FF: target field
  \2 $N$: persistent array size
  \2 $\vec a, \vec b \in \FF^N$: initial and final arrays
  \2 $c_a, c_b$: commitments to arrays with randomness $r_a, r_b$
  \2 $R$: a witness relation $R(c_a, c_b, x, \vec a, \vec b, w, r_a, r_b)$ for a CP-SNARK
    \3 $x$ extra instance material
    \3 $w$ extra witness material
    \3 $\vec a$ represents the initial state of a RAM that $R$ makes $A$
    accesses to; $\vec b$ is the final state.
    \2 $A$: number of accesses 
  \2 $[A]$: $\{0, \dots, A-1\}$
  \2 hashes
    \3 root hash: $H_r(k, \vec x) = \prod_i (x_i - k)$
      \4 used for ``permutation arguments''
      \4 used to hash multisets to scalars
    \3 coefficient hash: $H_c(k, \vec x) = \sum_{i=0} k^ix_i$
      \4 used to hash vectors to scalars
\1 coalesce the memory footprint
  \2 goal: from the $N$ entries, isolate up to $A$ that $R$ will be able to
  touch
    \3 route the remaining entries from $\vec a$ to $\vec b$ as cheaply as
       possible
  \2 witness: $\vec c$: $A$ index-value pairs of $\vec a$ including all touched entries.
  \2 witness: $\vec d$: similar, with $\vec b$ (same indices and order).
  \2 challenge $\alpha$
  \2 compute $\vec a'$, defined by $a'_i = H_c(\alpha, (a_i, i))$ and $\vec b'$ similarly
    \3 free (linear)
  \2 compute $\vec c'$, a length $N$ vector
    \3 First $A$ entries: root-hashes, similarly from $\vec c$.
      \4 $v + \alpha i$
      \4 This would be $A$ constraints, but it's shared with the main
      transcript's coefficient hashes, so it's free.
    \3 For the rest, hashes of other $\vec a$ entries, in original order (witness)
  \2 compute $\vec d'$ similarly, from $\vec d$ and $\vec b$.
    \3 Free, as above.
  \2 challenge $\beta$
  \2 equate last $N-A$ entries of $\vec c'$ and $\vec d'$.
    \3 free (linear)
  \2 permutation arguments for $\vec a', \vec c'$ and $\vec b', \vec d'$ (key $\beta$)
    \3 \csPerN{4}
  \2 outputs
    \3 $\vec c$ initial index-value pairs
    \3 $\vec d$ final index-value pairs
\1 build the main transcript
  \2 goal: define access-order and index-order transcripts, and ensure they're
  permutations of one another
  \2 $T$: number of entries ($3A$)
  \2 entries: (value, idx, time, write?, create?)
    \2 create is a bit that is set if this is an access writing from the initial persistent RAM
    \2 times are in $[A+1]$, not $[3A]$
      \3 accesses from $R$ gets times in $[A]$, as standard
      \3 all initial writes to have time 0
      \3 all final reads have time $A$
  \2 denote the transcript $\vec e$
  \2 witness: $\vec f$: the transcript ordered on index and then time
  \2 challenge $\alpha$
  \2 compute $\vec e'$ with element-wise coefficient hash over $\vec e$ keyed on $\alpha$.
    \3 The computations is something like $v + \alpha i + k_2 \alpha^2 k_3 \alpha^3 + k_4 \alpha^4$
    \3 The $k$'s are all fixed
    \3 \csPerT{1}
    \3 Furthermore, the single non-linear product $\alpha i$ is shared with the
    hashes needed for coalescing.
  \2 compute $\vec f'$ similarly from $\vec f$
    \3 \csPerT{4} for hashing
    \3 the hash ensures well-formedness for booleans
  \2 challenge $\beta$
  \2 permutation argument for $\vec e', \vec f'$ (key $\beta$)
    \3 \csPerT{2}
  \2 three properties remain to be checked
    \3 transcript is grouped by index
    \3 within that, sorted by time
    \3 read-over-write semantics
      \4 with no writing to fresh indices
\1 checking the index-order transcript
  \2 goal: check array semantics (modulo residual range checks)
  \2 Refer to one entry as $(v, i, t, w, c)$ and the next as $(v', i', t', w', c')$
  \2 Recall that the ranges of each value are ensured by hashing
    \3 $i \in [N]$
    \3 $t \in [A+1]$
    \3 $w, c \in \zo$
  \2 rules:
    \3 start with a create
    \3 $t$ grows or next $c=1$
    \3 $v$ is constant or next $w=1$
    \3 $i$ is constant or next $c=1$
  \2 constraints:
    \3 $c_0 = 1$
    \3 $(v'-v)(w'-1)=0$
    \3 $(i'-i)(c'-1)=0$
    \3 compute $\Delta_t = (1-c')(t'-t)$
    \3 cost: \csPerT{3}
      \4 and range check
      \4 and one for $c_0 = 1$
    \3 we must show (next section) that each $\Delta_t \in [A+1]$
\1 range check
  \2 goal: check that each $\Delta_t$ is in $[A+1]$
  \2 let $\vec g$ denote all $\Delta_t$s concatenated to $[A+1]$ (length $4A+1$)
  \2 witness $\vec h$: sort $\vec g$
    \3 check that each delta is 0 or 1. \csPerA{4}
  \2 challenge $\alpha$
  \2 permutation argument for $\vec g, \vec h$ (key $\alpha$)
    \3 \csPerA{8}
\1 Accounting
  \2 two verifier challenges ($\alpha,\beta$)
  \2 \csPerN{4}
  \2 \csPerA{42}
    \3 The cost of a linear scan (per element scanned) is $3$ by my reckoning
    \3 So if $A < 42/3 = 14$, you should do linear scans instead.
\1 Large but sparse memories
  \2 What if you want $\vec a$ and $\vec b$ to be index-value pairs, where the
  indices are from a large space (rather than $[N]$)?
  \2 I think this system is very close to supporting that
    \3 perhaps with an extra \csPerN{2}?
    \3 there are some weird semantic issues
      \4 what is the API for writing to fresh addresses?
      \4 is it different than writing to an occupied address?
      \4 can you read from a fresh address?
      \4 can you test for freshness?
\end{outline}
